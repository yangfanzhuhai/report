
\subsection{iBus System}
The actual arrival time of buses on a route at any given bus stop for the selected day are made available by the London Buses iBus system.

Currently, the routes were selected as the first of the New Routemaster (New Bus for London). Data will be published once a week \cite{buses_performance_data}. 

This data is stored in comma-separated values (CSV) format. There are two potential uses for this data. 
\begin{itemize}
	\item We can integrate it into the arrivals table to improve the precision of the arrivals data. Since the current entries in the arrivals table contain estimated bus arrival times, the integrated of the iBus data which contains real bus arrival times will likely boost the precision of the data in arrival table.
	\item We can compare the predicted delays with the iBus data for performance evaluation. 
\end{itemize}

\section{Additional Background Materials}
Here's a list of the potential points to be covered in the final report.

\begin{itemize}
	\item Geocoding data
    \item Possible analysis methodologies such as regression
    \item Literature review on similar research done in other areas of the world
\end{itemize}

% \section{Other Available Data}
% \subsection{Greater London Data}
% \subsection{Geocoding}

% \section{Analysis Methodologies}
% % \todo[inline]{I am not really sure what methods to use for analysis}
% \subsection{Regression}
% \subsection{Statistical Analysis}

% \section{Literature}
% \subsection{TFL Journey Planner}
% The website’s Journey Planner and homepage
% status board have also been redesigned and
% integrated so customers planning their trip can
% see immediately if their route is likely to be
% affected by upgrade work or other disruptions.
% The Journey Planner will now automatically
% generate cycling and walking routes, giving
% people a wider range of options for their journey \cite{tfl_annual_report_13/14}

% \subsection{Google Maps}
% \subsection{City Mapper London}
