%!TEX root = report.tex
\chapter{Delay Data Service}

\section{Overview}
To enable structured query of the predictions generated, we created a REST API service for the prediction data. We hope to make these endpoints freely available for other developers to use.

\section{Django Framework}
\par We used Django Framework\cite{django_framework} for the backend of this project. It is a high-level Python Web framework that encourages rapid development and clean, pragmatic design. We chose it for the following advantages:

\begin{itemize}
  \item It stores databases schema as data models\cite{django_model}, and allows for quick database migration. This makes deployment to the production server easy by using the given migration command \cite{django_migrations}.
  \item Django handles user authentication securely \cite{django_user_auth}, which saves work for development on this part.
  \item It has a nice compatible REST framework\cite{django_rest}, which supports REST API routing\cite{django_rest_routing} and query parameter parsing. Additionally, it offers a built-in user interface that displays the query result with pagination\cite{django_rest_pagination}.
  \item Django has a Debug mode that display constructive error messages, make the development much easier.
  \item It is written in Python, which is a preferred language for its simple and short syntax.
\end{itemize}

\section{API Endpoints}
\begin{itemize}
  \item address of the endpoint
  \item parameters supplied
  \item return result format
\end{itemize}

\subsection{Historical \& Current}
\subsection{Reference}
\subsection{Arrival}



\section{Summary of data service}
Met the first objective by supplying data on bus travel time predictions
