%!TEX root = report.tex
\chapter{Delay Data Service}
\label{ch:data_service}
\section{Overview}
\par To enable structured query of the predictions generated, we created a REST API service for the prediction data. We hope to make these endpoints freely available for other developers to use.

\section{Django Framework}
\par We used Django Framework\cite{django_framework} for the backend of this project. It is a high-level Python Web framework that encourages rapid development and clean, pragmatic design. We chose it for the following advantages:

\begin{itemize}
  \item It stores databases schema as data models\cite{django_model}, and allows for quick database migration. This makes deployment to the production server easy by using the given migration command \cite{django_migrations}.
  \item Django handles user authentication securely \cite{django_user_auth}, which saves work for development on this part.
  \item It has a nice compatible REST framework\cite{django_rest}, which supports REST API routing\cite{django_rest_routing} and query parameter parsing. Additionally, it offers a built-in user interface that displays the query result with pagination\cite{django_rest_pagination}.
  \item Django has a Debug mode that display constructive error messages, make the development much easier.
  \item It is written in Python, which is a preferred language for its simple and short syntax.
\end{itemize}

\section{API Endpoints}
\par We built the backend using Django framework, and exposed the following 3 endpoints for users to access data.

\subsection{Historical \& Current Timetables}
\par This API endpoint is built to provide information on how long the bus on a given route at a given hour of the given day will take to reach the future stops from a given starting stop.

\subsubsection{Query URL \& Parameters}
\par The base URL to access the historical and current timetables is \url{http://delay.doc.ic.ac.uk:5000/predictions/?}.

\par The following parameters are required to obtain a specific set of predictions:
\begin{itemize}
  \item \textbf{day} Day of the week
  \item \textbf{hour} Hour of the day [0 - 23]
  \item \textbf{route} Route
  \item \textbf{run} Route direction
  \item \textbf{naptan\_atco} The \gls{naptan} code for the starting downstream stops in the route
\end{itemize}

\par For example, the URL to retrieve historical and current bus travel time predictions for route 360 for downstream stops starting with \gls{naptan} code 490016263E at hour 15 on Tuesday is \url{http://delay.doc.ic.ac.uk:5000/predictions/?day=Tuesday&hour=15&route=360&run=2&naptan_atco=490016263E}.

\subsubsection{Result Format}
\par The result of the API query is a JSON list of bus stops, following the given route sequences in the given route direction, starting with the given stop.

\par Each bus stop entry contains information on the bus stop, as well as the bus travel time in seconds. The following four fields are the most important ones:

\begin{itemize}
  \item \textbf{average\_travel\_time} Historical average bus travel time from the previous stop to the current stop.
  \item \textbf{cumulative\_travel\_time} Historical average bus travel time from the given starting stop to the current stop.
  \item \textbf{average\_travel\_time} Current average bus travel time from the previous stop to the current stop.
  \item \textbf{cumulative\_travel\_time} Current average bus travel time from the given starting stop to the current stop.
\end{itemize}

\subsection{Reference Timetable}
\begin{itemize}
  \item address of the endpoint
  \item parameters supplied
  \item return result format
\end{itemize}
\url{http://delay.doc.ic.ac.uk:5000/tfl_timetable/?day=Tuesday&hour=15&route=360&run=2&naptan_atco=490016263E}
\subsection{Arrival}

\section{Summary of data service}
Met the first objective by supplying data on bus travel time predictions
