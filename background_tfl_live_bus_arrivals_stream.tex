%!TEX root = report.tex
\subsection{Live Bus Arrivals API Stream}
\par The Live Bus Arrivals API Stream provides the predicted time until a bus is expected to arrive at a stop. These predictions are available for the next 30 minutes at any point in time. For example, at 9am, the stream will provide predicted bus arrivals up to 9.30am on the same day. This data is refreshed every 30 seconds \cite{live_bus_api_documentation}.

\par The base URL is \url{http://countdown.api.tfl.gov.uk/interfaces/ura/stream_V1}. In order to collect bus arrival data for analysis, we supplied the following parameters which specify the fields returned by the API.

\subsubsection{Parameters supplied}
\begin{itemize}
	\item \textit{StopID} This is the alphanumeric identifier of a bus stop. It is also known as stop\_code\_lbsl.
 	\item \textit{LineName} This is the route number that is displayed on the front of the bus on any publicity advertising the route.
  \item \textit{VehicleID} The unique identifier of the vehicle.
  \item \textit{TripID} The identifer of the specific trip that the prediction is for.
  \item \textit{EstimatedTime} This is the predicted time of arrival for the vehicle at a specific stop.
  \item \textit{ExpireTime} This is the time at which the corresponding prediction is no longer valid and should stop being displayed.
\end{itemize}

\par The resulting URL is \sloppy \url{http://countdown.api.tfl.gov.uk/interfaces/ura/stream_V1?ReturnList=StopID,LineName,VehicleID,TripID,EstimatedTime,ExpireTime}.

\par Each data entry contains an estimated arrival time for each bus journey at a given bus stop. This estimated arrival time is stored in the database via an UPDATE statement, which ensures that only the latest estimated arrival times per journey per bus stop are stored. To avoid having data being overwritten, only the entries that have been changed in the recent 10 minutes can be updated. This has been achieved through running a daemon process written in python on a virtual host. Table \ref{table:delay_arrivals_schema} shows the schema for arrivals table.

\begin{table}
\centering
\begin{tabular}{@{}llr@{}} \toprule
Column Name & Type & Default \\ \midrule
id(Primary) & int(11) & Auto Increment \\
stop\_code\_lbsl & varchar(64) &  \\
route & varchar(64) &  \\
vehicle\_id & varchar(64) & \\
trip\_id & varchar(64) & \\
arrival\_date & date &  \\
arrival\_time & timestamp & NULL \\
expire\_time & timestamp & NULL \\
recorded\_time & timestamp & Current Timestamp \\ \bottomrule
\end{tabular}
\caption{delay\_arrivals Table Schema}
\label{table:delay_arrivals_schema}
\end{table}

\subsubsection{Assumption}
We assume that the actual bus arrival time is the midpoint between the last estimated arrival time, and the system time when the clear signal (\textit{ExpireTime} = 0) is received.