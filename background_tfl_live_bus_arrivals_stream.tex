%!TEX root = report.tex
\subsection{Live Bus Arrivals API}
\par The Live Bus Arrivals API provides the predicted time until a bus is expected to arrive at a stop. This \acrshort{api} is designed to enable application developers to subscribe to live bus information and use this data to develop innovative services\cite{live_bus_api_documentation}.

\par The predictions are generated from London buses locations, tracked using a combination of \acrfull{gps}, roadside transponders, gyro-meters to recognise orientation and turns, and software to match all of the information accurately to a point on a street. Next, the bus location information is transmitted back to a control centre which then works out the predicted bus journey time to reach each downstream stop, given typical journey times at that time of the day \cite{quora_generate_countdown}. These arrival predictions are available for the next 30 minutes from the current time. For example, at 9am, the \acrshort{api} will provide predicted bus arrivals up to 9.30am on the same day. This data is refreshed every 30 seconds \cite{live_bus_api_documentation}.

\par This \acrshort{api} is controlled via a number of different HTTP requests and parameters. A request is structured as follows: \\
\url{http://server/virtualDirectory/type/version?HTTP parameters}

\subsubsection{Data Types}

\par This \acrshort{api} service provides two types of request to users:

\begin{itemize}
  \item \texttt{instant}. The client polls the server and receives the response in a single message.
  \item \texttt{stream}. The client opens a persistent connection to the server. The server continually serves data to the client until the connection is terminated.
\end{itemize}

\par The data source for both instant and streaming requests is consistent, ensuring that the data provided to the public remains the same. We used this \acrshort{api} as a source for bus arrival times. See details in Section \ref{sec:collecting_arrival_times}.

