%!TEX root = report.tex
\chapter{Delay Data Service}
\label{ch:data_service}
\section{Overview}
\par To enable structured query of the predictions generated, we created a REST API service for the prediction data. We hope to make these endpoints freely available for other developers to use.

\section{Django Framework}
\par We used Django Framework\cite{django_framework} for the backend of this project. It is a high-level Python Web framework that encourages rapid development and clean, pragmatic design. We chose it for the following advantages:

\begin{itemize}
  \item It stores databases schema as data models\cite{django_model}, and allows for quick database migration. This makes deployment to the production server easy by using the given migration command \cite{django_migrations}.
  % \item Django handles user authentication securely \cite{django_user_auth}, which saves work for development on this part.
  \item It has a nice compatible REST framework\cite{django_rest}, which supports REST API routing\cite{django_rest_routing} and query parameter parsing. Additionally, it offers a built-in user interface that displays the query result with pagination\cite{django_rest_pagination}.
  \item Django has a Debug mode that display constructive error messages, make the development much easier.
  \item It is written in Python, which is our preferred language for its simple and short syntax.
\end{itemize}

\section{Deployment Setup}
\subsection{Python Virtual Environment}
\par In order to manage our Python project requirements neatly and make deployment simple, we set up virtual environment\cite{virtualenv} \cite{virtualenvwrapper} to manage the project requirements. This helps to separate the Python packages installed for different projects as well.

\subsection{Web Server}
We chose to use Nginx\cite{nginx} as the web server and Gunicorn\cite{gunicorn} as the Python Web Server Gateway Interface HTTP Server.

Nginx was chosen over Apache Web Server as it was more widely used with Django and Gunicorn setup, therefore there was more support articles available online for reference \cite{nginx_gunicorn_django}.

\section{API Endpoints}
\todo[inline]{Should I put screenshots for endpoints?}
\par We built the backend using Django framework and Django REST framework, and exposed the following 3 endpoints for users to access data. The base URL for all endpoints is \url{http://delay.doc.ic.ac.uk:5000}.
