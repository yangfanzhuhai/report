%!TEX root = report.tex

\subsection{Reference Timetable}
\par This API endpoint provides information on the \acrshort{tfl} official bus travel time for the buses on a given route at a given hour of the given day of week to reach downstream stops from a given starting stop.

\subsubsection{Query URL \& Parameters}
\par The base URL to access the reference timetable is \url{http://delay.doc.ic.ac.uk:5000/tfl_timetable/?}.The required parameters are the same for the predictions endpoint, described in Section \ref{sec:predictions_params}. For example, the URL to retrieve reference timetable entries for route 360 for downstream stops starting with \gls{naptan} code 490016263E at hour 15 on Tuesday is \url{http://delay.doc.ic.ac.uk:5000/tfl_timetable/?day=Tuesday&hour=15&route=360&run=2&naptan_atco=490016263E}.

\subsubsection{Result Format}
\par The result of this API query is similar to that of the prediction endpoint. It consists of a JSON list of bus stops, ordered by the given route sequences. Each bus stop entry contains information on the bus stop. Additionally, the \texttt{average\_travel\_time} field indicates the official bus travel time from the previous stop to the current stop in seconds, whereas the \texttt{cumulative\_travel\_time} field shows the official bus travel time from the givien starting stop to the current stop in seconds.

\subsubsection{Backend Computation}
The backend processes each request as such:

\begin{itemize}
  \item Search the \acrshort{tfl} timetable in databases for the entries for the given route, run and arrival hour.
  \item Filter the entries for the given day of the week, and the future stops of the given stop.
  \item Compute the cumulative travel time for the bus to reach each downstream stop from the given stop.
  \item Return the result in JSON format.
\end{itemize}


