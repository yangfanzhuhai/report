%!TEX root = report.tex
\section{Transport for London Open Data}

Open data is defined as data which can be used, re-used and re-distributed freely by anyone - subject only at most to the requirement to attribute and share-alike. There may be some charge, usually no more than the cost of reproduction \cite{open_data_def}.

\par It is \acrshort{tfl}'s strategy to provide free and open data. It began in 2007 using embedded widgets\cite{open_data_start}. In 2015, there are around 30 feeds and \acrshort{api}s available \cite{open_data}.

\par There are 3 forms of data:

\begin{itemize}
  \item static data files which rarely change,
  \item feeds that refreshed at regular intervals,
  \item and \acrshort{api}s that enable a query to receive a bespoke response, depending on the parameters supplied.
\end{itemize}

Over 5,000 developers have registered for the open data\cite{open_data}, and around 200 travel apps are powered by it \cite{tfl_annual_report_13/14}.
