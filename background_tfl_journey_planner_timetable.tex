%!TEX root = report.tex
\subsection{Journey Planner Bus Timetables}
In order to calculate the delays in bus arrival times, we need the official bus travel times between stops for reference. This data was extracted from the Journey Planner Bus Timetables\cite{open_data_feeds_description} as part of the TFL Open Data. The timetables contians information on bus schedules including stops, routes, departures times, departure frequencies, operational notes, as well as the days on which the services run.

The timetables uses the XML Schema Language\cite{xml} format, with the schema defined in TransXChange\cite{transxchange}, the UK nationwide standard for exchanging bus schedules and related data. We used the General schema version 2.1\cite{transxchange_downloads_and_schema}\cite{transxchange_schema_2.1_xsd} for this project.

\subsubsection{Data Structure}
The TransXchange model has seven basic concepts\cite{transxchange_schema_guide}:
\todo[inline]{insert definitions here}
\begin{enumerate}
  \item \textit{Service},
  \item \textit{Registration},
  \item \textit{Operator},
  \item \textit{Route} specifies an ordered list of \textit{StopPoints}.
  \item \textit{StopPoint} contains reusable declarations of the stops used by the routes and journey patterns of the schedule. All StopPointRef instances elsewhere in a document are resolved against the contents of the StopPoints element.
  \item \textit{JourneyPattern} specifies an ordered list of links between the \textit{StopPoints}, giving
relative times between each stop.
  \item \textit{VehicleJourney} specifies the list of stops at specific absolute passing times.
\end{enumerate}
<<<<<<< Updated upstream
\todo[inline]{Need more detailed explanation. Walk through an example with xml code}
\par Each xml file contains bus schedule information for a route. For each day of the week, there are predefined number of VehicleJourneys running the give route throughout the day. We used the VehicleJourneys as a starting point to retrieve the departure time of the actual vehicle from the terminal. We then retrieve the corresponding JourneyPattern, and obtained the travel time between each neighbouring stops on the route for the given vehicle journey, to compute the cumulative travel times throughout the route.

The above computation was performed on each xml file to generate the actual arrival time and travel time for each vehicle trip at each stop in the route throughout the day. The results of this computation was stored in the delay\_tfl\_timetable table(Table \ref{table:delay_tfl_timetable}).

\begin{table}
\centering
\begin{tabular}{@{}llp{6cm}@{}} \toprule
Column Name & Type & Comments\\ \midrule
id(Primary Key) & int(11)  & Auto Increment\\ [0.4cm]
route & varchar(64) & The bus route \\ [0.4cm]
day & varchar(32) & The day of week for the vehicle journey \\ [0.4cm]
run & int(11) & The route direction \\ [0.4cm]
sequence & int (11) & The sequence of the bus stop in the route \\ [0.4cm]
stop\_name & varchar(64) & The name of the bus stop \\ [0.4cm]
naptan\_atco & varchar(64) & The national identifier of the bus stop \\ [0.4cm]
arrival\_time & datetime(6) & The expected arrival time for the given vehicle at the current bus stop \\ [0.4cm]
travel\_time & int(11) & The travel time in seconds from the previous stop in the route to the current stop \\ [0.4cm]
cumulative\_travel\_time & int(11) & The travel time  in seconds from the terminal to the current stop \\ [0.4cm]
departure\_time\_from\_origin & datetime(6) & The departure time of the given vehicle from the terminal \\
\bottomrule
\end{tabular}
\caption{delay\_tfl\_timetable Table Schema}
\label{table:delay_tfl_timetable}
\end{table}
=======
>>>>>>> Stashed changes
