%!TEX root = report.tex
\subsection{Journey Planner Bus Timetables}
In order to calculate the delays in bus arrival times, we need the official bus travel times between stops for reference. This data was extracted from the Journey Planner Bus Timetables\cite{open_data_feeds_description} as part of the TFL Open Data. The timetables contians information on bus schedules including stops, routes, departures times, departure frequencies, operational notes, as well as the days on which the services run.

The timetables uses the XML Schema Language\cite{xml} format, with the schema defined in TransXChange\cite{transxchange}, the UK nationwide standard for exchanging bus schedules and related data. We used the General schema version 2.1\cite{transxchange_downloads_and_schema}\cite{transxchange_schema_2.1_xsd} for this project.

\subsubsection{Data Structure}
The TransXchange model has seven basic concepts\cite{transxchange_schema_guide}:
\todo[inline]{insert definitions here}
\begin{enumerate}
  \item \textit{Service},
  \item \textit{Registration},
  \item \textit{Operator},
  \item \textit{Route} specifies an ordered list of \textit{StopPoints}.
  \item \textit{StopPoint} contains reusable declarations of the stops used by the routes and journey patterns of the schedule. All StopPointRef instances elsewhere in a document are resolved against the contents of the StopPoints element.
  \item \textit{JourneyPattern} specifies an ordered list of links between the \textit{StopPoints}, giving
relative times between each stop.
  \item \textit{VehicleJourney} specifies the list of stops at specific absolute passing times.
\end{enumerate}

