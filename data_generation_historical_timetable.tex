%!TEX root = report.tex
\section{Generating Historical Timetable}
\label{sec:historical_timetable}
\par The historical timetable is generated from the current timetables. We processed the current timetable log by computing the average bus travel time by the start stop, end stop, day of the week, and hour of the day. This computation is performed weekly on Sunday midnight when the server is less busy to update the \texttt{delay\_historical\_timetable}. The following SQL script details the update task.

\begin{verbatim}
TRUNCATE delay_timetable_updated;

INSERT INTO delay_timetable_updated
(start_stop, end_stop, day, hour, average_travel_time)
(
  SELECT start_stop, end_stop, day, hour,
         AVG(average_travel_time) AS average_travel_time
  FROM delay_current_timetable_log
  GROUP BY start_stop, end_stop, day, hour
);

INSERT INTO delay_timetable
(start_stop, end_stop, day, hour, average_travel_time)
SELECT start_stop, end_stop, day, hour, average_travel_time
FROM delay_timetable_updated
ON DUPLICATE KEY UPDATE
delay_timetable.average_travel_time =
delay_timetable_updated.average_travel_time;
\end{verbatim}


\section{Summary}
\par The data collection and generation forms is the key stage to produce the reference, current, and historical timetables. After we crafted and tested the scripts and procedures to update the timetables, we proceeded to build the data service API. This is discussed in the next chapter.
