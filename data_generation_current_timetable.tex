%!TEX root = report.tex
\section{Generating Current Timetable}
\label{sec:current_timetable_generation}

\subsection{Collecting Bus Arrival Times}
\label{sec:collecting_arrival_times}

\subsubsection{Building the Query URL}
\par We collected bus arrival data for analysis from the live bus arrivals API. The base URL used in this project was \url{http://countdown.api.tfl.gov.uk/interfaces/ura/instant_V1}.

\par We supplied the following parameters which specify the fields returned by the \acrshort{api}.

\begin{itemize}
  \item \textit{StopID.} The alphanumeric identifier of a bus stop. It is also known as stop\_code\_lbsl.
  \item \textit{LineName.} The route number that is displayed on the front of the bus on any publicity advertising the route.
  \item \textit{DirectionID.} This could be number 1 or 2. Number 1 indicates an outbound journey, and number 2 indicates an inbound journey.
  \item \textit{VehicleID.} The unique identifier of the vehicle.
  \item \textit{TripID.} The identifier of the specific trip that the prediction is for.
  \item \textit{EstimatedTime.} The predicted time of arrival for the vehicle at a specific stop. It is in the form of UTC as per Unix epoch in milliseconds. Divide by 1000 to convert to UTC Epoch.
  \item \textit{ExpireTime.} The time at which the corresponding prediction is no longer valid and should stop being displayed.
\end{itemize}

\par The resulting query URL is \sloppy \url{http://countdown.api.tfl.gov.uk/interfaces/ura/instant_V1?ReturnList=StopID,LineName,DirectionID,VehicleID,TripID,EstimatedTime,ExpireTime}.

\subsubsection{Storing Arrival Times}
\par The TfL Live Bus Arrival Feed is updated every 30 seconds to give a more accurate predictions of the bus arrival times. We send an HTTP request to the above URL every 30 seconds.

\par Each data entry in the return result contains an estimated arrival time for each bus journey at a given bus stop. We assume that the actual bus arrival time is the midpoint between the last estimated arrival time, and the system time when the clear signal (\textit{ExpireTime} = 0) is received.

\par Since sometimes the clear signal is lost for certain entries, we assumed the actual bus arrival time is the latest recorded estimated arrival time, when there are no more updates 15 minutes after the expire time. This means that we have not received any new updates for the given bus at a given stop 15 minutes after the last estimated arrival time.

\par As we would like to only store the actual bus arrival times, we keep a local copy of the current query result using the Python \texttt{pickle} module\cite{pickle}, and only update the databases when the most arrival time for the given bus at the given stop has expired for more than 15 minutes.

\par Here are the detailed steps:

\begin{enumerate}
  \item Load the local arrivals objects if there exists a copy.
  \item Pull the new TfL arrivals predictions.
  \item Update the local arrivals objects with the new predictions.
  \item For arrival entries that have expired for more than 15 minutes, remove these entries from the local copy, and store them in the \texttt{delay\_arrivals} table in the databases (Schema shown in Table \ref{table:delay_arrivals_schema}).
\end{enumerate}

The above steps were implemented in a Python script. The interesting part of the code for step 4 is shown below. Each run of the steps takes approximately 10 to 15 seconds. We re-run the script 15 seconds after the previous run finishes.
\label{sec:arrivals_daemon}

\begin{verbatim}
def insert_arrivals_to_db(conn, arrival):
    cur = conn.cursor()
    sql_insert = ("INSERT INTO current_arrivals "
                  "(stop_code_lbsl, route, run, vehicle_id, trip_id, "
                  "arrival_time, expire_time, arrival_date) "
                  "VALUES (%s, %s, %s, %s, %s, %s, %s, DATE(%s)) ")

    # Filter out the entries in the cache with 0 expire time
    # or have not been updated for the past 15 minutes. We assume
    # these were the actual arrival times for the buses.
    expired = [v for v in arrival.values()
               if (v[6] is None or
                   (v[6] + datetime.timedelta(minutes=15)
                    <= datetime.datetime.now()))]

    # Filter out the recently updated entries in the cache.
    arrival = dict((k, v) for k, v in arrival.items()
                   if (v[6] is not None and
                       (v[6] + datetime.timedelta(minutes=15)
                        > datetime.datetime.now())
                       )
                   )
    # Insert the expired arrivals entries to the databases,
    # and return the recently updated arrivals entries to
    # store in local cache.
    for line in expired:
        cur.execute(sql_insert, line[0:8])
    return arrival
\end{verbatim}


\subsection{Generating Bus Sequences and Neighbouring Stops}
\label{sec:bus_stop_locations_routes}
\par We stored the distinct bus stops in Table \texttt{delay\_stops} (Schema in Table \ref{table:delay_stops}). We imported the bus routes data introduced in Section \ref{sec:bus_sequence} into the \texttt{delay\_bus\_sequences} table (Table \ref{table:delay_bus_sequences}). Every entry contains information on the route name, route direction, and the sequences of stops in the route. As the sequence information for some routes have gaps, we preprocessed the data by updating the sequence number of the following stops to fill up the gaps. This preprocessing step was done after a few examples verified with the up-to-date TfL routes in the TfL Journey Planner.

\par In order to find out the average travel time between any pair of neighbouring stops, we needed a list of all the neighbouring stops serving by various routes for reference. We extracted this information from the bus routes data using a simple Python script shown below. We then imported the \acrshort{csv}s into the \texttt{delay\_neighbours} table (Table \ref{table:delay_neighbours}).

\begin{verbatim}
with open('generated/neighbouring.csv', 'w') as out:
    writer = csv.writer(out)
    writer.writerow(['route', 'start_stop_code_lbsl',
                     'end_stop_code_lbsl'])
    for entry in data.values():
        # Sort the bus sequences by route, run, and sequence
        seq = sorted(entry.items(), key=lambda x: x[0])

        # Zip the sequence with itself by 1 position off to
        # obtain neighbouring pairs in tuple form
        for current, nxt in zip(seq, seq[1:]):
            writer.writerow([current[1]['Route'],
                            current[1]['Stop_Code_LBSL'],
                            nxt[1]['Stop_Code_LBSL']])
\end{verbatim}

\par In the sample entries shown in Table \ref{table:sample_neighbours_view}, we can see that there are three different routes serving between stop 10002 and 11469. When calculating the average travel time between these two stops at a given hour, we used all bus trip information for these three routes.

\begin{table}
\centering
\begin{tabular}{@{}llrr@{}} \toprule
id & route & start\_stop & end\_stop \\ \midrule
18433 & 30 & 10002 & 11469 \\
44878 & N19 & 10002 & 11469 \\
8653 & 19 & 10002 & 11469 \\ \bottomrule
\end{tabular}
\caption{Sample data in delay\_neighbours Table}
\label{table:sample_neighbours_view}
\end{table}

\subsection{Generating the Current Average Travel Time Between Neighbouring Stops}
\par To generate the current average travel time, we first isolated the arrival times collected in the recent one hour. The SQL statement is shown below.

\begin{verbatim}
TRUNCATE arrivals_last_hour;

INSERT INTO arrivals_last_hour
SELECT *
FROM delay_arrivals
WHERE recorded_time >= DATE_SUB(NOW(),INTERVAL 1 HOUR);
\end{verbatim}

\par We then performed the following steps on the last 1 hour arrivals data:

\begin{enumerate}
  \item For each bus travelling between each pair of the neighbouring bus stop,
  \begin{enumerate}
    \item Find out the arrival times for the same bus at the start stop and the end stop of the neighbouring pair.
    \item Calculate the difference of these two arrival times, and save it as one entry of the travel time. See SQL statement below.
  \end{enumerate}
\begin{verbatim}
TRUNCATE current_travel_time_log;

INSERT INTO current_travel_time_log
SELECT t1.stop_code_lbsl start_stop,
       t2.stop_code_lbsl end_stop,
       t1.route route,
       t1.trip_id trip_id,
       t1.vehicle_id vehicle_id,
       t1.arrival_time start_time,
       t2.arrival_time end_time,
       TIME_TO_SEC(TIMEDIFF(t2.arrival_time,
                            t1.arrival_time)) travel_time,
       t2.arrival_date `date`,
       DAYNAME(t2.arrival_time) day,
       HOUR(t2.arrival_time) hour
FROM delay_neighbours AS neighbours
INNER JOIN arrivals_last_hour AS t1
  ON t1.stop_code_lbsl = neighbours.start_stop
  AND t1.route = neighbours.route
INNER JOIN arrivals_last_hour AS t2
  ON t2.stop_code_lbsl = neighbours.end_stop
  AND t2.route = neighbours.route
  AND t1.trip_id = t2.trip_id
  AND t1.vehicle_id = t2.vehicle_id;
\end{verbatim}
  \item Compute the average travel time of all bus trips took place between every pair of neighbouring stops, and save it in the current timetable.

\begin{verbatim}
TRUNCATE delay_current_timetable_new;

INSERT INTO delay_current_timetable_new
(start_stop, end_stop, average_travel_time, `date`, day, hour)
SELECT start_stop, end_stop,
       TRUNCATE(AVG(travel_time), 1) average_travel_time,
       `date`, day, hour
FROM current_travel_time_log
GROUP BY start_stop, end_stop;
\end{verbatim}

  \item Save the current day of week, and hour of the day in the timetable for reference.
\end {enumerate}

\par We ran the above steps every hour to refresh the \texttt{delay\_current\_timetable} (Schema in Table \ref{table:delay_current_timetable}) . Before every update, we stored the current timetable into \texttt{delay\_current\_timetable\_log} table for generating the historical timetable.


% \subsection{Negative Travel Time Filter}
% \par In the travel\_time\_log generated, there were trips between two neighbouring stops with negative travel times, such as the entry shown in Table \ref{table:travel_time_log_negative}.

% \begin{table}
% \centering
% \begin{tabular}{@{}lllllr@{}} \toprule
% Start Stop & End Stop & Route & Start Time & End Time & Travel Time(sec) \\ \midrule
% 9326 & 15552 & W13 & 14:53:18 & 14:52:14 & -64 \\ \bottomrule
% \end{tabular}
% \caption{Travel Time Log Entry with Negative Travel Time}
% \label{table:travel_time_log_negative}
% \end{table}

% \par This was because when we performed the join of the arrivals table, there was a more recent update on the arrival times for the start stop, whereas the arrival times of end stop had not been updated. In Table \ref{table:negative_travel_time_explained}, we observed that the Recorded Time for the end stop 9326 was more recent than that of the end stop. As a result, the arrival time for the end stop was earlier than the start stop, causing the travel time to be negative.

% \begin{table}
% \centering
% \begin{tabular}{@{}lllllr@{}} \toprule
% Stop Code & Route & Vehicle ID & Trip ID & Arrival Time & Recorded Time\\ \midrule
% 9326 & W13 & 18685 & 135229 &  14:53:18 & 14:48:25 \\ [0.4cm]
% 15552 & W13 & 18685 & 135229 & 14:52:14 & 14:44:02 \\ \bottomrule
% \end{tabular}
% \caption{Arrivals Entries to Explain Negative Travel Time}
% \label{table:negative_travel_time_explained}
% \end{table}

% \par We filtered out these negative values before calculating the average travel time between neighbouring stops.
