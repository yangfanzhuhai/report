\chapter{Background}

% \section{Terminology}
% \todo[inline]{insert glossary here?}
% \begin{itemize}
% 	\item \textit{BusID} refers to 
%     \item \textit{TripID} refers to ... unique during all trips in a day.
% \end{itemize}

\section{London Bus Network}

\par The bus network in London is one of the largest and most accessible in the world. It is carrying a staggering
number of passengers, with more than 2.4
billion journeys in 2013/14, which was more than any year since 1959 \cite{tfl_annual_report_13/14}. 

\par On an average day between 2005 and 2010, about 14\% of the trips made by London residents were by bus \cite{tfl_ltds}. They spent on average 14 minutes per day on these bus trips.

\par There are currently 19,345 bus stops, and 680 routes served by 8,765 buses daily in London\cite{bus_stop_locations_routes}.

% number of people that use apps to plan journey or pick the bus to take

\subsection{Bus Network Performance}

\par TfL published the following figures in the second quarter 2014/2015 buses performance data \cite{buses_performance_report}.

\par For the high frequency services, the average scheduled wait was 4.86 minutes, the average excess wait was 0.94 minutes, and the average actual wait was 5.80 minutes. While passengers could expect the buses to come within 10 minutes 83.4\% of the time, there was 15.1\% chance of waiting for 10-20 minutes, 1.3\% chance of waiting for 20-30 minutes, and 0.2\% chance of waiting for more than 30 minutes.

\par For the low frequency services, 87\% of the buses services were on time, and 11.4\% were 5-15 minutes late. 

\par For the night buses, 84.5\% of the services were on time. The average excess wait was 0.68 minutes.

\par The bus arrivals might be affected by traffic congestion, staff availability, and engineering problems or mechanical breakdown \cite{buses_performance_data}.

