%!TEX root = report.tex
\chapter{Concept Design}
\section{Objectives}
 there is a need to improve the prediction of bus travel time downstream from location of last observation in mixed traffic operations. Methods are required for making predictions in normal traffic conditions, as well as in conditions in which a temporary lane closure is experienced due to a variety of reasons, such as incidents and road improvement activities

% MODELS FOR PREDICTING BUS DELAYS - ResearchGate. Available from: http://www.researchgate.net/publication/239438109_MODELS_FOR_PREDICTING_BUS_DELAYS [accessed May 28, 2015].
We try to find a solution for theses two problems by providing a service of bus travel time predictions, and a demo application to show case the use of such a data service.

\section{Overview}
\par \acrshort{tfl} releases average travel time between stops. However, this data is too general and sometimes not accurate.

\par The general idea to predict delays is that the average travel time between 2 stops depends mainly on the day of the week and the hour.

\section{Historical Timetable}
\par We collect historical bus arrival tiems, and culculate the average travel time between 2 neighbouring bus stops by all routes, on given day and hour

\section{Current Timetable}
\par We collect the past 1 hour of bus arrival times, and calculate the average travel time between neighbouring bus stops.

\section{Reference Timtable}
\par Get the official average travel time from journey planner timetable

\section{Assumption}
\par The historical timetable should converge to reference timetable.

\par When the current timetable is very different from reference timetable, there's delay.
