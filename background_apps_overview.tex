%!TEX root = report.tex
\section{Current Travel Applications}
\par Currently, the most popular travel applications powered by the \acrshort{tfl} Open Data include \acrshort{tfl} Journey Planner, Google Maps (Figure \ref{fig:google_maps_tfl_data}), and Citymapper London.

\begin{figure}
\centering
\includegraphics[width=1\textwidth]{figures/google_maps_tfl_data.png}
\caption{\label{fig:google_maps_tfl_data} Google Maps Using TfL Bus Arrival Data}
\end{figure}

\par Given the departing location, destination, as well as departure time, these applications can provide suggested routes and travel times. Users can further customise their desired journey by specifying the desired walking distance, and accessibility requirements. Moreover, the \acrshort{tfl} Journey Planner and homepage
status board were redesigned and integrated so customers planning their trip can see immediately if their route is likely to be affected by upgrade work or other disruptions \cite{tfl_annual_report_13/14}, with a textual warning message shown.

\par However, currently, there are no applications that give predictions on travel times and warnings on potential delays. The estimated travel time shown in apps is extracted from the \acrshort{tfl} Journey Planner Bus Timetables. This information does not capture the real time delays according to instant traffic conditions.
